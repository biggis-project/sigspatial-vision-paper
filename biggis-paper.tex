\documentclass{sig-alternate-05-2015}

\usepackage{xcolor}
\usepackage{pifont}
\usepackage{paralist} % inparaenum support

\newcommand{\quadrat}{\ding{110}}%


\begin{document}
% No indent of paragraph
\parindent 0pt

% Copyright
\setcopyright{acmcopyright}
%\setcopyright{acmlicensed}
%\setcopyright{rightsretained}
%\setcopyright{usgov}
%\setcopyright{usgovmixed}
%\setcopyright{cagov}
%\setcopyright{cagovmixed}


% DOI
\doi{10.475/123_4}

% ISBN
%\isbn{123-4567-24-567/08/06}

%Conference
\conferenceinfo{ACM SigSpatial '16}{October 31 - Thursday November 3, 2016, San Francisco Bay Area, California, USA}

%\acmPrice{\$15.00}

%
% --- Author Metadata here ---
%\conferenceinfo{WOODSTOCK}{'97 El Paso, Texas USA}
%\CopyrightYear{2007} % Allows default copyright year (20XX) to be over-ridden - IF NEED BE.
%\crdata{0-12345-67-8/90/01}  % Allows default copyright data (0-89791-88-6/97/05) to be over-ridden - IF NEED BE.
% --- End of Author Metadata ---
%\title{BigGIS: A new generation predictive and prescriptive Geographic Information System}
\title{BigGIS: A Predictive and Prescriptive Geographic Information System based on High-Dimensional Geo-Temporal Data Structures ("Vision Paper")}
%\titlenote{(Produces the permission block, and
%copyright information). For use with
%SIG-ALTERNATE.CLS. Supported by ACM.}}
%\subtitle{[Extended Abstract]
%\titlenote{A full version of this paper is available as
%\textit{Author's Guide to Preparing ACM SIG Proceedings Using
%\LaTeX$2_\epsilon$\ and BibTeX} at
%\texttt{www.acm.org/eaddress.htm}}}
%
% You need the command \numberofauthors to handle the 'placement
% and alignment' of the authors beneath the title.
%
% For aesthetic reasons, we recommend 'three authors at a time'
% i.e. three 'name/affiliation blocks' be placed beneath the title.
%
% NOTE: You are NOT restricted in how many 'rows' of
% "name/affiliations" may appear. We just ask that you restrict
% the number of 'columns' to three.
%
% Because of the available 'opening page real-estate'
% we ask you to refrain from putting more than six authors
% (two rows with three columns) beneath the article title.
% More than six makes the first-page appear very cluttered indeed.
%
% Use the \alignauthor commands to handle the names
% and affiliations for an 'aesthetic maximum' of six authors.
% Add names, affiliations, addresses for
% the seventh etc. author(s) as the argument for the
% \additionalauthors command.
% These 'additional authors' will be output/set for you
% without further effort on your part as the last section in
% the body of your article BEFORE References or any Appendices.

\numberofauthors{8} %  in this sample file, there are a *total*
% of EIGHT authors. SIX appear on the 'first-page' (for formatting
% reasons) and the remaining two appear in the \additionalauthors section.
%
\author{
% You can go ahead and credit any number of authors here,
% e.g. one 'row of three' or two rows (consisting of one row of three
% and a second row of one, two or three).
%
% The command \alignauthor (no curly braces needed) should
% precede each author name, affiliation/snail-mail address and
% e-mail address. Additionally, tag each line of
% affiliation/address with \affaddr, and tag the
% e-mail address with \email.
%
% 1st. author
\alignauthor
Firstname Lastname\titlenote{Maybe put address, e-mail here if allowed}\\
       \affaddr{University of Applied Sciences Karlsruhe}\\
       \affaddr{Moltkestr. 30}\\
       \affaddr{Karlsruhe, Germany}\\
       \email{firstname.lastname@hs-karlsruhe.de}
% 2nd. author
\alignauthor
Firstname Lastname\titlenote{Maybe put address, e-mail here if allowed}\\
       \affaddr{FZI Research Center for Information Technology}\\
       \affaddr{Haid-und-Neu-Str. 10-14}\\
       \affaddr{Karlsruhe, Germany}\\
       \email{lastname@fzi.de}
% 3rd. author
\alignauthor
Firstname Lastname\titlenote{Maybe put address, e-mail here if allowed}\\
       \affaddr{Data Analysis and Visualization Group}\\
       \affaddr{University of Konstanz}\\
       \affaddr{Konstanz, Germany}\\
       \email{firstname.lastname@uni-konstanz.de}
%\and  % use '\and' if you need 'another row' of author names
%%% 4th. author
%\alignauthor
%Firstname Lastname\\
%       \affaddr{FZI Research Center for Information Technology}\\
%       \affaddr{Haid-und-Neu-Str. 10-14}\\
%       \affaddr{Karlsruhe, Germany}\\
%       \email{lastname@fzi.de}
%%% 5th. author
%\alignauthor
%Firstname Lastname\\
%       \affaddr{Data Analysis and Visualization Group}\\
%       \affaddr{University of Konstanz}\\
%       \affaddr{Konstanz, Germany}\\
%       \email{firstname.lastname@uni-konstanz.de}
%%% 6th. author
%\alignauthor
%Firstname Lastname\\
%       \affaddr{Data Analysis and Visualization Group}\\
%       \affaddr{University of Konstanz}\\
%       \affaddr{Konstanz, Germany}\\
%       \email{firstname.lastname@uni-konstanz.de}
}
% There's nothing stopping you putting the seventh, eighth, etc.
% author on the opening page (as the 'third row') but we ask,
% for aesthetic reasons that you place these 'additional authors'
% in the \additional authors block, viz.
\additionalauthors{\textcolor{red}{ToDO@all} John Smith (FZI,
email: {\texttt{jsmith@fzi.de}}), John Smith (FZI,
email: {\texttt{jsmith@fzi.de}}), John Smith (Konstanz,
email: {\texttt{jsmith@konstanz.de}}), John Smith (Konstanz,
email: {\texttt{jsmith@konstanz.de}}).}
%\date{30 July 1999}
% Just remember to make sure that the TOTAL number of authors
% is the number that will appear on the first page PLUS the
% number that will appear in the \additionalauthors section.

\maketitle
\begin{abstract}
\textcolor{red}{ToDO@Patrick} 
%(Geospatial data has always been Big Data. However today's geographic information systems reach their limits due to rapidly growing geo-related data volumes. Especially in real-time processing, these systems do not provide a sufficient design. New technologies and architectures are emerging from the context of Big Data and gain increasing importance. However, dealing with a variety of different data and uncertainty still remains a crucial task.)
\end{abstract}


%
% The code below should be generated by the tool at
% http://dl.acm.org/ccs.cfm
% Please copy and paste the code instead of the example below. 
%
%\begin{CCSXML}
%<ccs2012>
% <concept>
%  <concept_id>10010520.10010553.10010562</concept_id>
%  <concept_desc>Computer systems organization~Embedded systems</concept_desc>
%  <concept_significance>500</concept_significance>
% </concept>
% <concept>
%  <concept_id>10010520.10010575.10010755</concept_id>
%  <concept_desc>Computer systems organization~Redundancy</concept_desc>
%  <concept_significance>300</concept_significance>
% </concept>
% <concept>
%  <concept_id>10010520.10010553.10010554</concept_id>
%  <concept_desc>Computer systems organization~Robotics</concept_desc>
%  <concept_significance>100</concept_significance>
% </concept>
% <concept>
%  <concept_id>10003033.10003083.10003095</concept_id>
%  <concept_desc>Networks~Network reliability</concept_desc>
%  <concept_significance>100</concept_significance>
% </concept>
%</ccs2012>  
%\end{CCSXML}
%
%\ccsdesc[500]{Computer systems organization~Embedded systems}
%\ccsdesc[300]{Computer systems organization~Redundancy}
%\ccsdesc{Computer systems organization~Robotics}
%\ccsdesc[100]{Networks~Network reliability}


%
% End generated code
%

%
%  Use this command to print the description
%
%\printccsdesc

% We no longer use \terms command
%\terms{Theory}

\keywords{big geospatial data processing; modelling uncertainty; semantics; visual analytics; knowledge generation}

\section{Introduction}
\label{sec:intro}
Geographic information systems (GIS) have long been used for geospatial data
analyses and visualizations to support the decision making process \cite{Crossland1995} in many
domains like civil planning, environment and nature protection or emergency
management. Thereby geospatial data have always been big data. Petabytes of
remotely sensed archival geodata (\textit{volume}) and a rapidly increasing
amount of real-time sensor data streams (\textit{velocity}) accelerate the need
for big data analytics in order to effectively model and efficiently process
complex geo-temporal problems. In the past, limited access to computing power
has been a bottleneck \cite{OGC2013}. However, in the era of cloud computing,
leveraging cloud-based resources is a widely adopted pattern (hardware level).
In addition, with the advent of big data analytics and its ever increasing scope
of application, performing massively parallel analytical tasks on large-scale
data at rest or data in motion is as well becoming a feasible approach shaping
the design of today's GIS (software-level). Although scaling out enables GIS to
tackle the aforementioned big data induced requirements, there are still two
major open issues. Firstly, dealing with varying data types across multiple data
sources (\textit{variety}) leading to data and schema heterogenity, e.g. to
describe locations, like addresses, relative spatial relationships or different
coordinates reference systems \cite{Frank.2016a}. Secondly, modelling the
inherent uncertainties in data (\textit{veracity}), e.g. real-world noise and
errorneous values due to the nature of the data collecting process
\cite{Xiang2016}. Both being crucial tasks in data management and analytics that
directly affect the information retrieval and decision making quality and
moreover the generated knowledge on human-side (\textit{value}). Delegating this
to only predefined and computed metrics often deprioritizes more important macro
aspects of the problem. Current approaches mainly
address high volume batch and high velocity stream analytics in their design of
a closed unified analytical GIS platform \cite{Thakur2015}. While the importance
of such systems to efficiently deal with large amount of data is obvious,
computers miss the creativity of human analysis to create hidden connections
between data and problem domain \cite{SSS+14a}.

In this paper, we present the vision of \textit{BigGIS}, a new generation 
predictive and prescriptive GIS, that
leverages big data analytics, semantic reasoning and visual analytics
methodologies. Our approach symbiotically combines system-side computation,
data storage and semantic reasoning capabilities with human-side perceptive
skills, cognitive reasoning and domain knowledge. Considering uncertainty to be
reciprocally related to generating new insights and consequently to knowledge,
we introduce a novel \textit{continuous refinement model} in Section
\ref{sec:crm} to gradually minimize the real-world noise and dissolve
heterogenity in data and metadata such that the information gain can be
maximized. Based on the well-established knowledge generation model for visual
analytics introduced in \cite{SSS+14a} this will on one hand allow to steadily
improve the analysis results by updating deployed machine learning models, on
the other hand to build the user's trust in these results by creating awareness
of underlying uncertainties and data provenance which is key for providing
meaningful predictive and prescriptive decision support in various fields
\cite{SSK+16a}. Our contribution lies in
\begin{inparaenum}[(1)]
  \item an \textit{integrated analytical pipeline approach} which includes
  \item \textit{semantic reasoning} as well as
  \item \textit{human-related knowledge extraction and generation} to process
  high volume, high velocity and high dimensional spatio-temporal data from
  unreliable and heterogeneous sources.
\end{inparaenum}

\section{Related Work}
\label{sec:related}

\textcolor{red}{ToDO@Patrick, maybe better to leave out completely or put into Section \ref{sec:chls}} Challenges related to the nature of big data, i.e. large, fast and diverse, lead to the evolution of new big data management and analytics architectures. Marz proposes the lambda architecture \cite{Marz2013}, a generic, scalable and fault-tolerant data processing architecture. By decomposing the problem into three layers, namely the batch layer, the speed layer, and the serving layer this architecture hybridly combines batch analytics on historic data and stream analytics on high-velocity streaming data to overcome eachs single weakenesses. While this design allows for adhoc analysis ability during batch runs, the processing logic has to be implemented twice. In a more recent approach, Kreps critizes the overall complexity of the lambda architecture and presents the kappa architecture \cite{Kreps2014}, which simplifies the systems' design by neglecting the batch layer. To replace batch processing, data is quickly fed through the streaming system. While this is beneficial for maintainability reasons, human-side domain knowledge is not considered to be fed into the system during runtime.
\\
\textcolor{red}{ToDO@all} (Demarcation of \cite{Peng2014}: Used BigGIS, 
	but not for a certain type of computation platform, more as a general 
	overview of how big data technologies can help answer GIS related questions. 
	Check doi in .bib-file for paper link)
\\
\textcolor{red}{ToDO@Patrick} Thakur et al. introduces PlanetSense, a real-time streaming and spatio-temporal analytics platform for gathering geo-spatial intelligence from open source data \cite{Thakur2015}. Based on the lambda architecture, this platform enriches large volumes of historic data by harvesting real-time data from various data sources on the fly, e.g. social media, or passive and participatory sensors (IoTs, traffic cameras, detectors) and allows for instant action providing an interactive environment for end users.
\\
\textcolor{red}{ToDO@Viliam} Plasmap\footnote{\url{https://plasmap.io/}} is a high performance geo-processing platform that provides a lightweight query language for high-performance location discovery based on OpenStreetMap. Plasmap differs from BigGIS ... TODO.
\\
\textcolor{red}{ToDO@Julian} (Analytics aka Magic) e.g. \cite{Shmueli2011} spatial predictive, prescriptive analytics, modelling uncertainty etc.
-> more methodological?
-> statistics for spatio-temporal data, copula uni münster, ?
\\
\textcolor{red}{ToDO@Matthias (Are there any GIS leveraging semantic web technologies? e.g. \textcolor{blue}{\url{http://bit.ly/27HjOuX}})} Preconditions for meaningful findings in GIS are accurate, consistent and complete data as input for analytical processes. However, sources of spatial data are distributed and quality of the data is varying, especially when considering uncertain data like volunteered geographic information and participatory sensing data. We address this challenge by a smart data integration approach~\cite{Frank.2016a} which is based on semantically described data sources and data transformation services. Smart web services dynamically compose workflows of data sources and data transformation services adopted to the requirements of different GISs based on the semantic meta data~\cite{Frank.2016b}.
\\
\textcolor{red}{ToDO@Manuel@Daniel (Are there any GIS leveraing VA methodologies? e.g. \textcolor{blue}{\url{http://bit.ly/1Rd8z1m}} or \textcolor{blue}{\url{http://bit.ly/1WGo7CP}} or \textcolor{blue}{\url{http://bit.ly/1U4EYZN}} 
)} Analyses are often performed in a descriptive, predictive or prescriptive way. While the descriptive analysis visualizes the status quo, predictive and prescriptive analysis focuses on future-oriented planning. As a result the underlying model and the visualization have to be tightly coupled in order for users to gain knowledge. Users have the possibility to interactively alter a model's parameters according to their knowledge, consequently the visualization adjusts to the model in a feedback-loop. Knowledge generation is one important research area where visual analytics is of great use~\cite{keim2010mastering}, especially when considering uncertainty of heterogeneous data from various data sources \cite{SSK+16a}. J\"ackle et al. present one possible visualization technique \cite{JSBK15} for data and uncertainties of large spatial datasets, which is crucial within use-cases where both facets are of importance for decision making.) 

%We group related work into geo-spatial big data-analytics platforms and handling uncertainty. The former covers works on big data enabled GIS platforms and discusses their conceptual designs. The latter gives an overview of fields of research with respect to uncertainty handling and knowledge generation.
%
%\subsection{Geo-Spatial Big Data-Analytics Platforms}
%\begin{itemize}
%	\item \textcolor{red}{ToDO@Patrick, maybe better to leave out completely or put into Section \ref{sec:chls}} Challenges related to the nature of big data, i.e. large, fast and diverse, lead to the evolution of new big data management and analytics architectures. Marz proposes the lambda architecture \cite{Marz2013} for generic, scalable and fault-tolerant data processing architecture. By decomposing the problem into three layers, namely the batch layer, the speed layer, and the serving layer this architecture hybridly combines batch analytics on historic data and stream analytics on high-velocity streaming data to overcome eachs single weakenesses. While this design allows for adhoc analysis ability during batch runs the processing logic has to be implemented twice. \\In a more recent approach, Kreps critizes the overall complexity of the lambda architecture and presents the kappa architecture \cite{Kreps2014}, which simplifies the systems design by neglecting the batch layer. To replace batch processing, data is quickly fed through the streaming system. While this is beneficial for maintainability reasons, human-side domain knowledge is not considered to be fed into the system during runtime.
%	\item \textcolor{red}{ToDO@all} (Demarcation of \cite{Peng2014}: Used BigGIS, 
%	but not for a certain type of computation platform, more as a general 
%	overview of how big data technologies can help answer GIS related questions. 
%	Check doi in .bib-file for paper link)
%	\item \textcolor{red}{ToDO@Patrick} Thakur et al. introduces PlanetSense, a real-time streaming and spatio-temporal analytics platform for gathering geo-spatial intelligence from open source data \cite{Thakur2015}. Based on the lambda architecture, this platform enriches large volumes of historic data by harvesting real-time data from various data sources on the fly, e.g. social media, or passive and participatory sensors (IoTs, traffic cameras, detectors) and allows for instant action providing an interactive environment for end users.
%	\item \textcolor{red}{ToDO@Viliam (commercialized products)} Plasmap\footnote{\url{https://plasmap.io/}} is a high-performance 
%  geo-processing platform that provides a lightweight query language for 
%  high-performance location discovery based on OpenStreetMap. Plasmap differs from BigGIS ... TODO.
%\end{itemize}
%
%%\subsection{Handling Uncertainty}
%
%\begin{itemize}
%	\item \textcolor{red}{ToDO@Julian} (Analytics aka Magic)
%	\item \textcolor{red}{ToDO@Matthias} (Semantics: Cognintive Apps, Linked APIs)
%	Preconditions for meaningful findings in geographic information systems are accurate, consistent and complete data as input for analytical processes. However, sources of spatial data are distributed and quality of the data is varying, especially when considering uncertain data like volunteered geographic information and participatory sensing data. We address this challenge by a smart data integration approach~\cite{Frank.2016a} which is based on semantically described data sources and data transformation services. Smart web services dynamically compose workflows of data sources and data transformation services adopted to the requirements of different geographic information systems based on the semantic meta data~\cite{Frank.2016b}.
%	\item \textcolor{red}{ToDO@Manuel@Daniel} (Visual Analytics: Knowledge generation Model for Visual Analytics \cite{SSS+14a}, Analyses are often performed in a descriptive, predictive or prescriptive way. While the descriptive analysis visualizes the status quo, predictive and prescriptive analysis focuses on future-oriented planning. As a result the underlying model and the visualization have to be tightly coupled in order for users to gain knowledge. Users have the possibility to interactively alter a model's parameters according to their knowledge, consequently the visualization adjusts to the model in a feedback-loop. Knowledge generation is one important research area where visual analytics is of great use~\cite{keim2010mastering}, especially when considering uncertainty of heterogeneous data from various data sources \cite{SSK+16a}. J\"ackle et al. present one possible visualization technique \cite{JSBK15} for data and uncertainties of large spatial datasets, which is crucial within use-cases where both facets are of importance for decision making. \\The adequate support of diverse users is another field of research where visual analytics methods are beneficial.) 
%\end{itemize}

\section{BigGIS Platform}
\label{sec:biggis}

\subsection{Continuous Refinement Model in BigGIS}
\label{sec:crm}
 We consider uncertainty to be reciprocally related to generating new insights and consequently knowledge. Thus modelling uncertainty in BigGIS is a crucial task. In this section, we briefly describe the continuous refinement model in BigGIS from a high-level perspective, consisting of an integrated analytics pipeline which blends analytics and semantic reasoning on system-side with knowledge extraction and generation on human-side, thereby modelling uncertainty to continuously refine results as shown in Figure \ref{fig:biggisworkflow}. \textcolor{red}{ToDO@all: Clearly define terms of \textit{user}, \textit{semantics}, \textit{expert knowledge}, see issue \#16}

\begin{figure}
\centering
	\includegraphics[width=\linewidth]{figures/biggis-workflow_v3}
	\caption{Continuous Refinement Model in BigGIS}
	\label{fig:biggisworkflow}
\end{figure}
%Generic integrated analytical pipeline approach of BigGIS platform showing general purpose Continuous Refinement Process. Overcoming the usability gap between raw data and feature extraction (information gain, insight) and handling uncertainty in acquired data through indirect and direct semantic-level and expert-level support leveraged by the Knowledge Generation Model for Visual Analytics is one main contribution of BigGIS. Indirect support results from defined Refinement Gates (R-Gates \textcolor{yellow}{\quadrat}).


\subsubsection{Integrated Analytics Pipeline}
The analytics pipeline is the core of the continuous refinement model. A key abstraction within this model are specific access points called \textit{refinement gates} (see yellow squares in Figure \ref{fig:biggisworkflow}). Refinement gates allow for semantics and external expert knowledge to enter the pipeline at arbitrary stages during analyses to continuously improve data management and analytics results, e.g. to support data preparation, to automatically deploy data transformation workflows, to provide expert domain knowledge in order to train machine learning models for pattern detection or to manipulate visualizations. 

\subsubsection{Semantic Reasoning}
%Locating all available data sources that are relevant for meaningful findings in analytical processes is hard to do when it has to be done manually. Semantic web technology helps to describe data sources semantically using widely-applied vocabularies. Furthermore, semantic reasoning enables machines to discover suitable sources even if they are described differently, provided that an appropriate ontology supports the system.
Locating all available data sources that are relevant for meaningful findings in analytical processes is hard to do when it has to be done manually. Semantic web technology helps to describe data sources semantically using widely-applied vocabularies. Furthermore, semantic reasoning enables machines to discover suitable sources even if they are described differently, providing a two-level support for users through what we call \textit{linked APIs} and \textit{cognitive apps}. The former abstracts away the users from manually composing data integration workflows to unify heterogeneous data sources by using an appropriate ontology that supports the system (direct semantic support). The latter is a flexible service that is aware of a situational context and capable of sharing this with other services (indirect semantic support).

\textcolor{red}{ToDO@Matthias: please rework this section, any further ideas?}

\subsubsection{Knowledge Extraction and Generation}
 The user is another relevant part in the continuous refinement model that is either provided with additional expert knowledge by another person or he himself is the expert in a specific field of application (direct expert knowledge). Overall, we see the continuous refinement process as a knowledge transfer from human (expert knowledge) to system which is reinforced by semantic reasoning. Thereby, human knowledge is introduced to the system that can contain additional domain specific information and constraints. By doing so, big data analytics can 
\begin{inparaenum}[(1)]
 	\item leverage the creativity of human analysis to be able to establish hidden connections between data and the problem domain and
	\item continuously refine the analyses quality and results.
\end{inparaenum}
The system intelligently learns from the provided external domain knowledge, such that it can reuse it in the future (indirect expert support). Thus, leading to an increasing likelihood of relevant findings by users during the course of exploration and eventually to generating new knowledge.
\textcolor{red}{ToDO@Manuel@Daniel: please rework this section, any further ideas?}
\textcolor{red}{ToDO@Julian: Analytical approach}



\subsubsection{Modelling Uncertainty}
\textcolor{red}{ToDO@Matthias: semantics-level}

When considering uncertain data like volunteered geographic information and participatory sensed data in geographic information systems, we also have to deal with uncertainty. To address this challenge, we apply semantic reasoning on the provenance information of data sources in order to infer a level of uncertainty that can be considered in the analytical processes.

\textcolor{red}{ToDO@Manuel@Daniel: please add text from VA perspective, how do we handle uncertainty?}\newline
\textcolor{red}{ToDO@Julian: please add text from Analytics perspective, how do we handle uncertainty?}

\subsection{Challenges}
\label{sec:chls}
We identify X major challenges... Firstly, ... Secondly, ... Thirdly, ...

\begin{itemize}
	\item \textcolor{red}{ToDO@all}: What challenges do we expect to have? According to the Subsections in Section \ref{sec:crm}
	\item Handling big data related requirements: volume, velocity, variety and veracity
	\item Volume and velocity are well-addressed requirements through highly scalable cloud-based architectures such as lambda architecture or kappa architecture. However, dealing with data and schema heterogenity and inherent uncertainty is an interesting field of research for geographic information systems that BigGIS addresses.  
	\item Dimension reduction for spatio-temporal data
	\item bias-variance trade-off / robustness
	\item runtime issues in real usage and transformation knowledge to user
\end{itemize}

\section{Use Cases}
\textcolor{red}{ToDO@all: please review and comment}
\\BigGIS will support decision making in multiple use cases that require processing of large and heterogeneous spatio-temporal data from unreliable sources. The prototype will be evaluated on three scenarios: 
\begin{inparaenum}[(1)]
	\item smart city and health, i.e. heat stress in urban areas,
	\item environmental management, i.e. spread of invasive species,
	\item emergency management, i.e. identification and dispersion in chemical accidents .
\end{inparaenum}
These scenarios represent diverse categories of application domains that each address varying big data related requirements. We assume that our proposed continuous refinement model in BigGIS allows tackling the requirements and helps generating meaningful knowledge. In brief, one example is the environmental management scenario, where farmers can provide real-time data (velocity) from differing sources (variety), e.g. private weather stations, photos of invasive species, or text messages about contaminated areas, though arriving with high uncertainty (veracity). The experts domain knowledge helps to train classifiers in BigGIS on already available labeled datasets (volume) that, in addition to further semantically described sources, helps conducting spatio-temporal statistics such as hot spot analyses to make better predictions on potentially jeopardized areas. Not only are farmers informed about the condition of their fields (descriptive) but also about risk potential of contamination (predictive), which lastly results in suggestions to perform certain counteractive measures (prescriptive).

\section{Discussion and Future Work}
\textcolor{red}{ToDO@all} (In the discussion part we discuss the advantages and disadvantages of our approach. We describe how we want to tackle the challenges addressed in Section \ref{sec:chls}. In the conclusion we again shortly summarize in our own words what we just presented and give insight about future work.)
We assume that our proposed continuous refinement model helps generating meaning knowledge...
%\end{document}  % This is where a 'short' article might terminate

%ACKNOWLEDGMENTS are optional
\section{Acknowledgements}
\label{sec:ack}
This research has been funded by the Federal Ministry of Education and Research of Germany (subsidy program IKT 2020 -- Forschung f\"ur Innovation).

%
% The following two commands are all you need in the
% initial runs of your .tex file to
% produce the bibliography for the citations in your paper.
\bibliographystyle{abbrv}
\bibliography{biggis-paper}  % sigproc.bib is the name of the Bibliography in this case
% You must have a proper ".bib" file
%  and remember to run:
% latex bibtex latex latex
% to resolve all references
%
% ACM needs 'a single self-contained file'!
%
%APPENDICES are optional
%\balancecolumns
%\appendix
%Appendix A


\end{document}
